% Options for packages loaded elsewhere
\PassOptionsToPackage{unicode}{hyperref}
\PassOptionsToPackage{hyphens}{url}
%
\documentclass[
]{article}
\usepackage{amsmath,amssymb}
\usepackage{lmodern}
\usepackage{iftex}
\ifPDFTeX
  \usepackage[T1]{fontenc}
  \usepackage[utf8]{inputenc}
  \usepackage{textcomp} % provide euro and other symbols
\else % if luatex or xetex
  \usepackage{unicode-math}
  \defaultfontfeatures{Scale=MatchLowercase}
  \defaultfontfeatures[\rmfamily]{Ligatures=TeX,Scale=1}
\fi
% Use upquote if available, for straight quotes in verbatim environments
\IfFileExists{upquote.sty}{\usepackage{upquote}}{}
\IfFileExists{microtype.sty}{% use microtype if available
  \usepackage[]{microtype}
  \UseMicrotypeSet[protrusion]{basicmath} % disable protrusion for tt fonts
}{}
\makeatletter
\@ifundefined{KOMAClassName}{% if non-KOMA class
  \IfFileExists{parskip.sty}{%
    \usepackage{parskip}
  }{% else
    \setlength{\parindent}{0pt}
    \setlength{\parskip}{6pt plus 2pt minus 1pt}}
}{% if KOMA class
  \KOMAoptions{parskip=half}}
\makeatother
\usepackage{xcolor}
\usepackage[margin=1in]{geometry}
\usepackage{color}
\usepackage{fancyvrb}
\newcommand{\VerbBar}{|}
\newcommand{\VERB}{\Verb[commandchars=\\\{\}]}
\DefineVerbatimEnvironment{Highlighting}{Verbatim}{commandchars=\\\{\}}
% Add ',fontsize=\small' for more characters per line
\usepackage{framed}
\definecolor{shadecolor}{RGB}{248,248,248}
\newenvironment{Shaded}{\begin{snugshade}}{\end{snugshade}}
\newcommand{\AlertTok}[1]{\textcolor[rgb]{0.94,0.16,0.16}{#1}}
\newcommand{\AnnotationTok}[1]{\textcolor[rgb]{0.56,0.35,0.01}{\textbf{\textit{#1}}}}
\newcommand{\AttributeTok}[1]{\textcolor[rgb]{0.77,0.63,0.00}{#1}}
\newcommand{\BaseNTok}[1]{\textcolor[rgb]{0.00,0.00,0.81}{#1}}
\newcommand{\BuiltInTok}[1]{#1}
\newcommand{\CharTok}[1]{\textcolor[rgb]{0.31,0.60,0.02}{#1}}
\newcommand{\CommentTok}[1]{\textcolor[rgb]{0.56,0.35,0.01}{\textit{#1}}}
\newcommand{\CommentVarTok}[1]{\textcolor[rgb]{0.56,0.35,0.01}{\textbf{\textit{#1}}}}
\newcommand{\ConstantTok}[1]{\textcolor[rgb]{0.00,0.00,0.00}{#1}}
\newcommand{\ControlFlowTok}[1]{\textcolor[rgb]{0.13,0.29,0.53}{\textbf{#1}}}
\newcommand{\DataTypeTok}[1]{\textcolor[rgb]{0.13,0.29,0.53}{#1}}
\newcommand{\DecValTok}[1]{\textcolor[rgb]{0.00,0.00,0.81}{#1}}
\newcommand{\DocumentationTok}[1]{\textcolor[rgb]{0.56,0.35,0.01}{\textbf{\textit{#1}}}}
\newcommand{\ErrorTok}[1]{\textcolor[rgb]{0.64,0.00,0.00}{\textbf{#1}}}
\newcommand{\ExtensionTok}[1]{#1}
\newcommand{\FloatTok}[1]{\textcolor[rgb]{0.00,0.00,0.81}{#1}}
\newcommand{\FunctionTok}[1]{\textcolor[rgb]{0.00,0.00,0.00}{#1}}
\newcommand{\ImportTok}[1]{#1}
\newcommand{\InformationTok}[1]{\textcolor[rgb]{0.56,0.35,0.01}{\textbf{\textit{#1}}}}
\newcommand{\KeywordTok}[1]{\textcolor[rgb]{0.13,0.29,0.53}{\textbf{#1}}}
\newcommand{\NormalTok}[1]{#1}
\newcommand{\OperatorTok}[1]{\textcolor[rgb]{0.81,0.36,0.00}{\textbf{#1}}}
\newcommand{\OtherTok}[1]{\textcolor[rgb]{0.56,0.35,0.01}{#1}}
\newcommand{\PreprocessorTok}[1]{\textcolor[rgb]{0.56,0.35,0.01}{\textit{#1}}}
\newcommand{\RegionMarkerTok}[1]{#1}
\newcommand{\SpecialCharTok}[1]{\textcolor[rgb]{0.00,0.00,0.00}{#1}}
\newcommand{\SpecialStringTok}[1]{\textcolor[rgb]{0.31,0.60,0.02}{#1}}
\newcommand{\StringTok}[1]{\textcolor[rgb]{0.31,0.60,0.02}{#1}}
\newcommand{\VariableTok}[1]{\textcolor[rgb]{0.00,0.00,0.00}{#1}}
\newcommand{\VerbatimStringTok}[1]{\textcolor[rgb]{0.31,0.60,0.02}{#1}}
\newcommand{\WarningTok}[1]{\textcolor[rgb]{0.56,0.35,0.01}{\textbf{\textit{#1}}}}
\usepackage{graphicx}
\makeatletter
\def\maxwidth{\ifdim\Gin@nat@width>\linewidth\linewidth\else\Gin@nat@width\fi}
\def\maxheight{\ifdim\Gin@nat@height>\textheight\textheight\else\Gin@nat@height\fi}
\makeatother
% Scale images if necessary, so that they will not overflow the page
% margins by default, and it is still possible to overwrite the defaults
% using explicit options in \includegraphics[width, height, ...]{}
\setkeys{Gin}{width=\maxwidth,height=\maxheight,keepaspectratio}
% Set default figure placement to htbp
\makeatletter
\def\fps@figure{htbp}
\makeatother
\setlength{\emergencystretch}{3em} % prevent overfull lines
\providecommand{\tightlist}{%
  \setlength{\itemsep}{0pt}\setlength{\parskip}{0pt}}
\setcounter{secnumdepth}{-\maxdimen} % remove section numbering
\ifLuaTeX
  \usepackage{selnolig}  % disable illegal ligatures
\fi
\IfFileExists{bookmark.sty}{\usepackage{bookmark}}{\usepackage{hyperref}}
\IfFileExists{xurl.sty}{\usepackage{xurl}}{} % add URL line breaks if available
\urlstyle{same} % disable monospaced font for URLs
\hypersetup{
  pdftitle={Actividad colaborativa en clase \textbar{} Base de datos de virus de NCBI},
  pdfauthor={Gabriel M},
  hidelinks,
  pdfcreator={LaTeX via pandoc}}

\title{Actividad colaborativa en clase \textbar{} Base de datos de virus
de NCBI}
\author{Gabriel M}
\date{2023-04-20}

\begin{document}
\maketitle

LIBRARIES

\begin{Shaded}
\begin{Highlighting}[]
\FunctionTok{library}\NormalTok{(}\StringTok{\textquotesingle{}seqinr\textquotesingle{}}\NormalTok{)}
\end{Highlighting}
\end{Shaded}

\begin{Shaded}
\begin{Highlighting}[]
\NormalTok{Zika }\OtherTok{\textless{}{-}} \FunctionTok{read.fasta}\NormalTok{(}\StringTok{\textquotesingle{}Zika.fasta\textquotesingle{}}\NormalTok{)}
\NormalTok{Corona\_2 }\OtherTok{\textless{}{-}} \FunctionTok{read.fasta}\NormalTok{(}\StringTok{\textquotesingle{}Corona\_2.fasta\textquotesingle{}}\NormalTok{)}
\NormalTok{Corona\_ME }\OtherTok{\textless{}{-}} \FunctionTok{read.fasta}\NormalTok{(}\StringTok{\textquotesingle{}Corona\_ME.fasta\textquotesingle{}}\NormalTok{)}
\NormalTok{Dengue }\OtherTok{\textless{}{-}} \FunctionTok{read.fasta}\NormalTok{(}\StringTok{\textquotesingle{}Dengue.fasta\textquotesingle{}}\NormalTok{)}
\NormalTok{Influenza\_A }\OtherTok{\textless{}{-}} \FunctionTok{read.fasta}\NormalTok{(}\StringTok{\textquotesingle{}Influenza\_A.fasta\textquotesingle{}}\NormalTok{)}
\NormalTok{Chikungunya }\OtherTok{\textless{}{-}} \FunctionTok{read.fasta}\NormalTok{(}\StringTok{\textquotesingle{}Chikungunya.fasta\textquotesingle{}}\NormalTok{)}
\end{Highlighting}
\end{Shaded}

\begin{enumerate}
\def\labelenumi{\arabic{enumi}.}
\tightlist
\item
  ¿Cuál es el tamaño de cada secuencia?
\end{enumerate}

\begin{Shaded}
\begin{Highlighting}[]
\NormalTok{tamaño }\OtherTok{\textless{}{-}} \ControlFlowTok{function}\NormalTok{(seq)\{}
  \FunctionTok{return}\NormalTok{(}\FunctionTok{nchar}\NormalTok{(seq))}
\NormalTok{\}}

\CommentTok{\# El tamaño de cada secuencia:}
\NormalTok{tamaño(Zika)}
\end{Highlighting}
\end{Shaded}

\begin{verbatim}
## NC_035889.1 
##       54148
\end{verbatim}

\begin{Shaded}
\begin{Highlighting}[]
\NormalTok{tamaño(Corona\_2)}
\end{Highlighting}
\end{Shaded}

\begin{verbatim}
## NC_045512.2 
##      149812
\end{verbatim}

\begin{Shaded}
\begin{Highlighting}[]
\NormalTok{tamaño(Corona\_ME)}
\end{Highlighting}
\end{Shaded}

\begin{verbatim}
## NC_019843.3 
##      150894
\end{verbatim}

\begin{Shaded}
\begin{Highlighting}[]
\NormalTok{tamaño(Dengue)}
\end{Highlighting}
\end{Shaded}

\begin{verbatim}
## NC_001477.1 
##       53782
\end{verbatim}

\begin{Shaded}
\begin{Highlighting}[]
\NormalTok{tamaño(Influenza\_A)}
\end{Highlighting}
\end{Shaded}

\begin{verbatim}
## CY064707.1 
##      11488
\end{verbatim}

\begin{Shaded}
\begin{Highlighting}[]
\NormalTok{tamaño(Chikungunya )}
\end{Highlighting}
\end{Shaded}

\begin{verbatim}
## EU037962.1 
##      59142
\end{verbatim}

2 - 3. ¿Cúal es la composición de nucleótidos de cada secuencia? - ¿Cuál
es el contenido de GC de cada virus?

\begin{Shaded}
\begin{Highlighting}[]
\NormalTok{nucleotide\_data }\OtherTok{\textless{}{-}} \ControlFlowTok{function}\NormalTok{(dna)\{}
  \FunctionTok{print}\NormalTok{(}\StringTok{\textquotesingle{}Tamaño de secuencia\textquotesingle{}}\NormalTok{)}
  \FunctionTok{print}\NormalTok{(}\FunctionTok{length}\NormalTok{(dna[[}\DecValTok{1}\NormalTok{]]))}
  \FunctionTok{print}\NormalTok{(}\StringTok{\textquotesingle{}Composición de nucleótidos\textquotesingle{}}\NormalTok{)}
  \FunctionTok{print}\NormalTok{(}\FunctionTok{count}\NormalTok{(dna[[}\DecValTok{1}\NormalTok{]],}\DecValTok{1}\NormalTok{))}
  \FunctionTok{print}\NormalTok{(}\StringTok{\textquotesingle{}Cantidad de GC en la secuencia\textquotesingle{}}\NormalTok{)}
  \FunctionTok{count}\NormalTok{(dna[[}\DecValTok{1}\NormalTok{]],}\DecValTok{2}\NormalTok{)[}\StringTok{\textquotesingle{}gc\textquotesingle{}}\NormalTok{]}
  
\NormalTok{\}}
\FunctionTok{print}\NormalTok{(}\StringTok{"Zika:"}\NormalTok{)}
\end{Highlighting}
\end{Shaded}

\begin{verbatim}
## [1] "Zika:"
\end{verbatim}

\begin{Shaded}
\begin{Highlighting}[]
\FunctionTok{nucleotide\_data}\NormalTok{(Zika)}
\end{Highlighting}
\end{Shaded}

\begin{verbatim}
## [1] "Tamaño de secuencia"
## [1] 10808
## [1] "Composición de nucleótidos"
## 
##    a    c    g    t 
## 2956 2383 3157 2312 
## [1] "Cantidad de GC en la secuencia"
\end{verbatim}

\begin{verbatim}
##  gc 
## 656
\end{verbatim}

\begin{Shaded}
\begin{Highlighting}[]
\FunctionTok{print}\NormalTok{(}\StringTok{"Corona\_2:"}\NormalTok{)}
\end{Highlighting}
\end{Shaded}

\begin{verbatim}
## [1] "Corona_2:"
\end{verbatim}

\begin{Shaded}
\begin{Highlighting}[]
\FunctionTok{nucleotide\_data}\NormalTok{(Corona\_2)}
\end{Highlighting}
\end{Shaded}

\begin{verbatim}
## [1] "Tamaño de secuencia"
## [1] 29903
## [1] "Composición de nucleótidos"
## 
##    a    c    g    t 
## 8954 5492 5863 9594 
## [1] "Cantidad de GC en la secuencia"
\end{verbatim}

\begin{verbatim}
##   gc 
## 1168
\end{verbatim}

\begin{Shaded}
\begin{Highlighting}[]
\FunctionTok{print}\NormalTok{(}\StringTok{"Corona\_ME:"}\NormalTok{)}
\end{Highlighting}
\end{Shaded}

\begin{verbatim}
## [1] "Corona_ME:"
\end{verbatim}

\begin{Shaded}
\begin{Highlighting}[]
\FunctionTok{nucleotide\_data}\NormalTok{(Corona\_ME)}
\end{Highlighting}
\end{Shaded}

\begin{verbatim}
## [1] "Tamaño de secuencia"
## [1] 30119
## [1] "Composición de nucleótidos"
## 
##    a    c    g    t 
## 7900 6116 6304 9799 
## [1] "Cantidad de GC en la secuencia"
\end{verbatim}

\begin{verbatim}
##   gc 
## 1490
\end{verbatim}

\begin{Shaded}
\begin{Highlighting}[]
\FunctionTok{print}\NormalTok{(}\StringTok{"Dengue:"}\NormalTok{)}
\end{Highlighting}
\end{Shaded}

\begin{verbatim}
## [1] "Dengue:"
\end{verbatim}

\begin{Shaded}
\begin{Highlighting}[]
\FunctionTok{nucleotide\_data}\NormalTok{(Dengue)}
\end{Highlighting}
\end{Shaded}

\begin{verbatim}
## [1] "Tamaño de secuencia"
## [1] 10735
## [1] "Composición de nucleótidos"
## 
##    a    c    g    t 
## 3426 2240 2770 2299 
## [1] "Cantidad de GC en la secuencia"
\end{verbatim}

\begin{verbatim}
##  gc 
## 500
\end{verbatim}

\begin{Shaded}
\begin{Highlighting}[]
\FunctionTok{print}\NormalTok{(}\StringTok{"Influenza\_A:"}\NormalTok{)}
\end{Highlighting}
\end{Shaded}

\begin{verbatim}
## [1] "Influenza_A:"
\end{verbatim}

\begin{Shaded}
\begin{Highlighting}[]
\FunctionTok{nucleotide\_data}\NormalTok{(Influenza\_A)}
\end{Highlighting}
\end{Shaded}

\begin{verbatim}
## [1] "Tamaño de secuencia"
## [1] 2293
## [1] "Composición de nucleótidos"
## 
##   a   c   g   t 
## 771 433 591 498 
## [1] "Cantidad de GC en la secuencia"
\end{verbatim}

\begin{verbatim}
## gc 
## 99
\end{verbatim}

\begin{Shaded}
\begin{Highlighting}[]
\FunctionTok{print}\NormalTok{(}\StringTok{"Chikungunya:"}\NormalTok{)}
\end{Highlighting}
\end{Shaded}

\begin{verbatim}
## [1] "Chikungunya:"
\end{verbatim}

\begin{Shaded}
\begin{Highlighting}[]
\FunctionTok{nucleotide\_data}\NormalTok{(Chikungunya)}
\end{Highlighting}
\end{Shaded}

\begin{verbatim}
## [1] "Tamaño de secuencia"
## [1] 11805
## [1] "Composición de nucleótidos"
## 
##    a    c    g    t 
## 3492 2950 2968 2385 
## [1] "Cantidad de GC en la secuencia"
\end{verbatim}

\begin{verbatim}
##  gc 
## 744
\end{verbatim}

\begin{enumerate}
\def\labelenumi{\arabic{enumi}.}
\setcounter{enumi}{3}
\tightlist
\item
  Crear una función para obtener la secuencia en complementaria e
  imprimirla por cada secuencia
\end{enumerate}

\begin{Shaded}
\begin{Highlighting}[]
\NormalTok{dnaToComplementary }\OtherTok{\textless{}{-}}\ControlFlowTok{function}\NormalTok{ (dna)\{}
\NormalTok{  complementaryDna }\OtherTok{\textless{}{-}}\StringTok{\textquotesingle{}\textquotesingle{}}
  \ControlFlowTok{for}\NormalTok{ (base }\ControlFlowTok{in} \DecValTok{1}\SpecialCharTok{:}\FunctionTok{nchar}\NormalTok{(dna))\{}
    \ControlFlowTok{if}\NormalTok{ (}\FunctionTok{substr}\NormalTok{(dna, base, base)}\SpecialCharTok{==} \StringTok{\textquotesingle{}T\textquotesingle{}}\NormalTok{)\{}
\NormalTok{      complementaryDna }\OtherTok{=} \FunctionTok{paste}\NormalTok{(complementaryDna,}\StringTok{\textquotesingle{}A\textquotesingle{}}\NormalTok{,}\AttributeTok{sep =} \StringTok{\textquotesingle{}\textquotesingle{}}\NormalTok{)}
\NormalTok{    \}}
    \ControlFlowTok{else} \ControlFlowTok{if}\NormalTok{ (}\FunctionTok{substr}\NormalTok{(dna, base, base)}\SpecialCharTok{==} \StringTok{\textquotesingle{}C\textquotesingle{}}\NormalTok{)\{}
\NormalTok{      complementaryDna }\OtherTok{=} \FunctionTok{paste}\NormalTok{(complementaryDna,}\StringTok{\textquotesingle{}G\textquotesingle{}}\NormalTok{, }\AttributeTok{sep =} \StringTok{\textquotesingle{}\textquotesingle{}}\NormalTok{)   }
\NormalTok{    \}}
   \ControlFlowTok{else} \ControlFlowTok{if}\NormalTok{ (}\FunctionTok{substr}\NormalTok{(dna, base, base)}\SpecialCharTok{==} \StringTok{\textquotesingle{}G\textquotesingle{}}\NormalTok{)\{}
\NormalTok{      complementaryDna }\OtherTok{=} \FunctionTok{paste}\NormalTok{(complementaryDna,}\StringTok{\textquotesingle{}C\textquotesingle{}}\NormalTok{, }\AttributeTok{sep=}\StringTok{\textquotesingle{}\textquotesingle{}}\NormalTok{)}
\NormalTok{    \}}
   \ControlFlowTok{else} \ControlFlowTok{if}\NormalTok{ (}\FunctionTok{substr}\NormalTok{(dna, base, base)}\SpecialCharTok{==} \StringTok{\textquotesingle{}A\textquotesingle{}}\NormalTok{)\{}
\NormalTok{      complementaryDna }\OtherTok{=} \FunctionTok{paste}\NormalTok{(complementaryDna,}\StringTok{\textquotesingle{}T\textquotesingle{}}\NormalTok{, }\AttributeTok{sep =} \StringTok{\textquotesingle{}\textquotesingle{}}\NormalTok{)   }
\NormalTok{    \}}
\NormalTok{  \}}
  \FunctionTok{return}\NormalTok{ (complementaryDna)}
\NormalTok{\}}
\end{Highlighting}
\end{Shaded}

\begin{enumerate}
\def\labelenumi{\arabic{enumi}.}
\setcounter{enumi}{4}
\tightlist
\item
  Crear una gráfica de resumen para comparar la composición de
  nucleótidos de las 5 secuencias.
\end{enumerate}

\begin{Shaded}
\begin{Highlighting}[]
\NormalTok{compara\_1 }\OtherTok{\textless{}{-}} \ControlFlowTok{function}\NormalTok{()\{}
  \CommentTok{\# par(mfrow=c(1,5))}
  
  \FunctionTok{barplot}\NormalTok{(}\FunctionTok{table}\NormalTok{(Zika),}\AttributeTok{col =} \DecValTok{1}\SpecialCharTok{:}\DecValTok{4}\NormalTok{,}\AttributeTok{main=}\StringTok{"Zika"}\NormalTok{,}\AttributeTok{xlab=}\StringTok{"Nucleotide"}\NormalTok{,}\AttributeTok{ylab =} \StringTok{"Instances of Nucleotide"}\NormalTok{)}
  \FunctionTok{barplot}\NormalTok{(}\FunctionTok{table}\NormalTok{(Corona\_2),}\AttributeTok{col =} \DecValTok{1}\SpecialCharTok{:}\DecValTok{4}\NormalTok{,}\AttributeTok{main=}\StringTok{"Corona\_2"}\NormalTok{,}\AttributeTok{xlab=}\StringTok{"Nucleotide"}\NormalTok{,}\AttributeTok{ylab =} \StringTok{"Instances of Nucleotide"}\NormalTok{)}
  \FunctionTok{barplot}\NormalTok{(}\FunctionTok{table}\NormalTok{(Corona\_ME),}\AttributeTok{col =} \DecValTok{1}\SpecialCharTok{:}\DecValTok{4}\NormalTok{,}\AttributeTok{main=}\StringTok{"Corona\_ME"}\NormalTok{,}\AttributeTok{xlab=}\StringTok{"Nucleotide"}\NormalTok{,}\AttributeTok{ylab =} \StringTok{"Instances of Nucleotide"}\NormalTok{)}
  \FunctionTok{barplot}\NormalTok{(}\FunctionTok{table}\NormalTok{(Dengue),}\AttributeTok{col =} \DecValTok{1}\SpecialCharTok{:}\DecValTok{4}\NormalTok{,}\AttributeTok{main=}\StringTok{"Dengue"}\NormalTok{,}\AttributeTok{xlab=}\StringTok{"Nucleotide"}\NormalTok{,}\AttributeTok{ylab =} \StringTok{"Instances of Nucleotide"}\NormalTok{)}
  \FunctionTok{barplot}\NormalTok{(}\FunctionTok{table}\NormalTok{(Influenza\_A),}\AttributeTok{col =} \DecValTok{1}\SpecialCharTok{:}\DecValTok{4}\NormalTok{,}\AttributeTok{main=}\StringTok{"Influenza\_A"}\NormalTok{,}\AttributeTok{xlab=}\StringTok{"Nucleotide"}\NormalTok{,}\AttributeTok{ylab =} \StringTok{"Instances of Nucleotide"}\NormalTok{)}
  \FunctionTok{barplot}\NormalTok{(}\FunctionTok{table}\NormalTok{(Chikungunya),}\AttributeTok{col =} \DecValTok{1}\SpecialCharTok{:}\DecValTok{4}\NormalTok{,}\AttributeTok{main=}\StringTok{"Chikungunya"}\NormalTok{,}\AttributeTok{xlab=}\StringTok{"Nucleotide"}\NormalTok{,}\AttributeTok{ylab =} \StringTok{"Instances of Nucleotide"}\NormalTok{)}
\NormalTok{\}}

\NormalTok{compara\_2 }\OtherTok{\textless{}{-}} \ControlFlowTok{function}\NormalTok{(seq1,seq2,seq3,seq4,seq5,seq6)\{}
  \CommentTok{\# par(mfrow=c(1,5))}
  \FunctionTok{barplot}\NormalTok{(}\FunctionTok{c}\NormalTok{(}\FunctionTok{count}\NormalTok{(seq1[[}\DecValTok{1}\NormalTok{]],}\DecValTok{2}\NormalTok{)),}\AttributeTok{col =} \DecValTok{1}\SpecialCharTok{:}\DecValTok{16}\NormalTok{,}\AttributeTok{main=}\StringTok{"Zika"}\NormalTok{,}\AttributeTok{xlab=}\StringTok{"Nucleotide Pairs"}\NormalTok{,}\AttributeTok{ylab =} \StringTok{"Instances of Nucleotide"}\NormalTok{)}
  \FunctionTok{barplot}\NormalTok{(}\FunctionTok{c}\NormalTok{(}\FunctionTok{count}\NormalTok{(seq2[[}\DecValTok{1}\NormalTok{]],}\DecValTok{2}\NormalTok{)),}\AttributeTok{col =} \DecValTok{1}\SpecialCharTok{:}\DecValTok{16}\NormalTok{,}\AttributeTok{main=}\StringTok{"Corona\_2"}\NormalTok{,}\AttributeTok{xlab=}\StringTok{"Nucleotide Pairs"}\NormalTok{,}\AttributeTok{ylab =} \StringTok{"Instances of Nucleotide"}\NormalTok{)}
  \FunctionTok{barplot}\NormalTok{(}\FunctionTok{c}\NormalTok{(}\FunctionTok{count}\NormalTok{(seq3[[}\DecValTok{1}\NormalTok{]],}\DecValTok{2}\NormalTok{)),}\AttributeTok{col =} \DecValTok{1}\SpecialCharTok{:}\DecValTok{16}\NormalTok{,}\AttributeTok{main=}\StringTok{"Corona\_ME"}\NormalTok{,}\AttributeTok{xlab=}\StringTok{"Nucleotide Pairs"}\NormalTok{,}\AttributeTok{ylab =} \StringTok{"Instances of Nucleotide"}\NormalTok{)}
  \FunctionTok{barplot}\NormalTok{(}\FunctionTok{c}\NormalTok{(}\FunctionTok{count}\NormalTok{(seq4[[}\DecValTok{1}\NormalTok{]],}\DecValTok{2}\NormalTok{)),}\AttributeTok{col =} \DecValTok{1}\SpecialCharTok{:}\DecValTok{16}\NormalTok{,}\AttributeTok{main=}\StringTok{"Dengue"}\NormalTok{,}\AttributeTok{xlab=}\StringTok{"Nucleotide Pairs"}\NormalTok{,}\AttributeTok{ylab =} \StringTok{"Instances of Nucleotide"}\NormalTok{)}
  \FunctionTok{barplot}\NormalTok{(}\FunctionTok{c}\NormalTok{(}\FunctionTok{count}\NormalTok{(seq5[[}\DecValTok{1}\NormalTok{]],}\DecValTok{2}\NormalTok{)),}\AttributeTok{col =} \DecValTok{1}\SpecialCharTok{:}\DecValTok{16}\NormalTok{,}\AttributeTok{main=}\StringTok{"Influenza\_A"}\NormalTok{,}\AttributeTok{xlab=}\StringTok{"Nucleotide Pairs"}\NormalTok{,}\AttributeTok{ylab =} \StringTok{"Instances of Nucleotide"}\NormalTok{)}
  \FunctionTok{barplot}\NormalTok{(}\FunctionTok{c}\NormalTok{(}\FunctionTok{count}\NormalTok{(seq6[[}\DecValTok{1}\NormalTok{]],}\DecValTok{2}\NormalTok{)),}\AttributeTok{col =} \DecValTok{1}\SpecialCharTok{:}\DecValTok{16}\NormalTok{,}\AttributeTok{main=}\StringTok{"Chikungunya"}\NormalTok{,}\AttributeTok{xlab=}\StringTok{"Nucleotide Pairs"}\NormalTok{,}\AttributeTok{ylab =} \StringTok{"Instances of Nucleotide"}\NormalTok{)}
\NormalTok{\}}
\FunctionTok{par}\NormalTok{(}\AttributeTok{mfrow=}\FunctionTok{c}\NormalTok{(}\DecValTok{1}\NormalTok{,}\DecValTok{6}\NormalTok{))}
\FunctionTok{compara\_1}\NormalTok{()}
\end{Highlighting}
\end{Shaded}

\includegraphics{Actividad-colaborativa-en-clase_Base-de-datos-de-virus-de-NCBI_files/figure-latex/unnamed-chunk-6-1.pdf}

\begin{Shaded}
\begin{Highlighting}[]
\FunctionTok{par}\NormalTok{(}\AttributeTok{mfrow=}\FunctionTok{c}\NormalTok{(}\DecValTok{1}\NormalTok{,}\DecValTok{1}\NormalTok{))}
\FunctionTok{compara\_1}\NormalTok{()}
\end{Highlighting}
\end{Shaded}

\includegraphics{Actividad-colaborativa-en-clase_Base-de-datos-de-virus-de-NCBI_files/figure-latex/unnamed-chunk-6-2.pdf}
\includegraphics{Actividad-colaborativa-en-clase_Base-de-datos-de-virus-de-NCBI_files/figure-latex/unnamed-chunk-6-3.pdf}
\includegraphics{Actividad-colaborativa-en-clase_Base-de-datos-de-virus-de-NCBI_files/figure-latex/unnamed-chunk-6-4.pdf}
\includegraphics{Actividad-colaborativa-en-clase_Base-de-datos-de-virus-de-NCBI_files/figure-latex/unnamed-chunk-6-5.pdf}
\includegraphics{Actividad-colaborativa-en-clase_Base-de-datos-de-virus-de-NCBI_files/figure-latex/unnamed-chunk-6-6.pdf}
\includegraphics{Actividad-colaborativa-en-clase_Base-de-datos-de-virus-de-NCBI_files/figure-latex/unnamed-chunk-6-7.pdf}

\begin{Shaded}
\begin{Highlighting}[]
\FunctionTok{par}\NormalTok{(}\AttributeTok{mfrow=}\FunctionTok{c}\NormalTok{(}\DecValTok{1}\NormalTok{,}\DecValTok{6}\NormalTok{))}
\FunctionTok{compara\_2}\NormalTok{(Zika,Corona\_2,Corona\_ME,Dengue,Influenza\_A,Chikungunya)}
\end{Highlighting}
\end{Shaded}

\includegraphics{Actividad-colaborativa-en-clase_Base-de-datos-de-virus-de-NCBI_files/figure-latex/unnamed-chunk-6-8.pdf}

\begin{Shaded}
\begin{Highlighting}[]
\FunctionTok{par}\NormalTok{(}\AttributeTok{mfrow=}\FunctionTok{c}\NormalTok{(}\DecValTok{1}\NormalTok{,}\DecValTok{1}\NormalTok{))}
\FunctionTok{compara\_2}\NormalTok{(Zika,Corona\_2,Corona\_ME,Dengue,Influenza\_A,Chikungunya)}
\end{Highlighting}
\end{Shaded}

\includegraphics{Actividad-colaborativa-en-clase_Base-de-datos-de-virus-de-NCBI_files/figure-latex/unnamed-chunk-6-9.pdf}
\includegraphics{Actividad-colaborativa-en-clase_Base-de-datos-de-virus-de-NCBI_files/figure-latex/unnamed-chunk-6-10.pdf}
\includegraphics{Actividad-colaborativa-en-clase_Base-de-datos-de-virus-de-NCBI_files/figure-latex/unnamed-chunk-6-11.pdf}
\includegraphics{Actividad-colaborativa-en-clase_Base-de-datos-de-virus-de-NCBI_files/figure-latex/unnamed-chunk-6-12.pdf}
\includegraphics{Actividad-colaborativa-en-clase_Base-de-datos-de-virus-de-NCBI_files/figure-latex/unnamed-chunk-6-13.pdf}
\includegraphics{Actividad-colaborativa-en-clase_Base-de-datos-de-virus-de-NCBI_files/figure-latex/unnamed-chunk-6-14.pdf}

\end{document}
