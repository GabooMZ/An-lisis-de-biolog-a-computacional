% Options for packages loaded elsewhere
\PassOptionsToPackage{unicode}{hyperref}
\PassOptionsToPackage{hyphens}{url}
%
\documentclass[
]{article}
\usepackage{amsmath,amssymb}
\usepackage{lmodern}
\usepackage{iftex}
\ifPDFTeX
  \usepackage[T1]{fontenc}
  \usepackage[utf8]{inputenc}
  \usepackage{textcomp} % provide euro and other symbols
\else % if luatex or xetex
  \usepackage{unicode-math}
  \defaultfontfeatures{Scale=MatchLowercase}
  \defaultfontfeatures[\rmfamily]{Ligatures=TeX,Scale=1}
\fi
% Use upquote if available, for straight quotes in verbatim environments
\IfFileExists{upquote.sty}{\usepackage{upquote}}{}
\IfFileExists{microtype.sty}{% use microtype if available
  \usepackage[]{microtype}
  \UseMicrotypeSet[protrusion]{basicmath} % disable protrusion for tt fonts
}{}
\makeatletter
\@ifundefined{KOMAClassName}{% if non-KOMA class
  \IfFileExists{parskip.sty}{%
    \usepackage{parskip}
  }{% else
    \setlength{\parindent}{0pt}
    \setlength{\parskip}{6pt plus 2pt minus 1pt}}
}{% if KOMA class
  \KOMAoptions{parskip=half}}
\makeatother
\usepackage{xcolor}
\usepackage[margin=1in]{geometry}
\usepackage{color}
\usepackage{fancyvrb}
\newcommand{\VerbBar}{|}
\newcommand{\VERB}{\Verb[commandchars=\\\{\}]}
\DefineVerbatimEnvironment{Highlighting}{Verbatim}{commandchars=\\\{\}}
% Add ',fontsize=\small' for more characters per line
\usepackage{framed}
\definecolor{shadecolor}{RGB}{248,248,248}
\newenvironment{Shaded}{\begin{snugshade}}{\end{snugshade}}
\newcommand{\AlertTok}[1]{\textcolor[rgb]{0.94,0.16,0.16}{#1}}
\newcommand{\AnnotationTok}[1]{\textcolor[rgb]{0.56,0.35,0.01}{\textbf{\textit{#1}}}}
\newcommand{\AttributeTok}[1]{\textcolor[rgb]{0.77,0.63,0.00}{#1}}
\newcommand{\BaseNTok}[1]{\textcolor[rgb]{0.00,0.00,0.81}{#1}}
\newcommand{\BuiltInTok}[1]{#1}
\newcommand{\CharTok}[1]{\textcolor[rgb]{0.31,0.60,0.02}{#1}}
\newcommand{\CommentTok}[1]{\textcolor[rgb]{0.56,0.35,0.01}{\textit{#1}}}
\newcommand{\CommentVarTok}[1]{\textcolor[rgb]{0.56,0.35,0.01}{\textbf{\textit{#1}}}}
\newcommand{\ConstantTok}[1]{\textcolor[rgb]{0.00,0.00,0.00}{#1}}
\newcommand{\ControlFlowTok}[1]{\textcolor[rgb]{0.13,0.29,0.53}{\textbf{#1}}}
\newcommand{\DataTypeTok}[1]{\textcolor[rgb]{0.13,0.29,0.53}{#1}}
\newcommand{\DecValTok}[1]{\textcolor[rgb]{0.00,0.00,0.81}{#1}}
\newcommand{\DocumentationTok}[1]{\textcolor[rgb]{0.56,0.35,0.01}{\textbf{\textit{#1}}}}
\newcommand{\ErrorTok}[1]{\textcolor[rgb]{0.64,0.00,0.00}{\textbf{#1}}}
\newcommand{\ExtensionTok}[1]{#1}
\newcommand{\FloatTok}[1]{\textcolor[rgb]{0.00,0.00,0.81}{#1}}
\newcommand{\FunctionTok}[1]{\textcolor[rgb]{0.00,0.00,0.00}{#1}}
\newcommand{\ImportTok}[1]{#1}
\newcommand{\InformationTok}[1]{\textcolor[rgb]{0.56,0.35,0.01}{\textbf{\textit{#1}}}}
\newcommand{\KeywordTok}[1]{\textcolor[rgb]{0.13,0.29,0.53}{\textbf{#1}}}
\newcommand{\NormalTok}[1]{#1}
\newcommand{\OperatorTok}[1]{\textcolor[rgb]{0.81,0.36,0.00}{\textbf{#1}}}
\newcommand{\OtherTok}[1]{\textcolor[rgb]{0.56,0.35,0.01}{#1}}
\newcommand{\PreprocessorTok}[1]{\textcolor[rgb]{0.56,0.35,0.01}{\textit{#1}}}
\newcommand{\RegionMarkerTok}[1]{#1}
\newcommand{\SpecialCharTok}[1]{\textcolor[rgb]{0.00,0.00,0.00}{#1}}
\newcommand{\SpecialStringTok}[1]{\textcolor[rgb]{0.31,0.60,0.02}{#1}}
\newcommand{\StringTok}[1]{\textcolor[rgb]{0.31,0.60,0.02}{#1}}
\newcommand{\VariableTok}[1]{\textcolor[rgb]{0.00,0.00,0.00}{#1}}
\newcommand{\VerbatimStringTok}[1]{\textcolor[rgb]{0.31,0.60,0.02}{#1}}
\newcommand{\WarningTok}[1]{\textcolor[rgb]{0.56,0.35,0.01}{\textbf{\textit{#1}}}}
\usepackage{graphicx}
\makeatletter
\def\maxwidth{\ifdim\Gin@nat@width>\linewidth\linewidth\else\Gin@nat@width\fi}
\def\maxheight{\ifdim\Gin@nat@height>\textheight\textheight\else\Gin@nat@height\fi}
\makeatother
% Scale images if necessary, so that they will not overflow the page
% margins by default, and it is still possible to overwrite the defaults
% using explicit options in \includegraphics[width, height, ...]{}
\setkeys{Gin}{width=\maxwidth,height=\maxheight,keepaspectratio}
% Set default figure placement to htbp
\makeatletter
\def\fps@figure{htbp}
\makeatother
\setlength{\emergencystretch}{3em} % prevent overfull lines
\providecommand{\tightlist}{%
  \setlength{\itemsep}{0pt}\setlength{\parskip}{0pt}}
\setcounter{secnumdepth}{-\maxdimen} % remove section numbering
\ifLuaTeX
  \usepackage{selnolig}  % disable illegal ligatures
\fi
\IfFileExists{bookmark.sty}{\usepackage{bookmark}}{\usepackage{hyperref}}
\IfFileExists{xurl.sty}{\usepackage{xurl}}{} % add URL line breaks if available
\urlstyle{same} % disable monospaced font for URLs
\hypersetup{
  pdftitle={Actividad colaborativa en clase \textbar{} Ejercicios R},
  pdfauthor={Gabriel M},
  hidelinks,
  pdfcreator={LaTeX via pandoc}}

\title{Actividad colaborativa en clase \textbar{} Ejercicios R}
\author{Gabriel M}
\date{2023-04-13}

\begin{document}
\maketitle

\begin{Shaded}
\begin{Highlighting}[]
\FunctionTok{library}\NormalTok{(}\StringTok{\textquotesingle{}stringr\textquotesingle{}}\NormalTok{)}
\FunctionTok{library}\NormalTok{(}\StringTok{\textquotesingle{}stringi\textquotesingle{}}\NormalTok{)}
\end{Highlighting}
\end{Shaded}

\begin{enumerate}
\def\labelenumi{\arabic{enumi}.}
\tightlist
\item
  Escribe una función que genere una secuencia aleatoria de DNA de
  tamaño ``n''.
\end{enumerate}

\begin{Shaded}
\begin{Highlighting}[]
\NormalTok{crearsecuenciaAdn }\OtherTok{\textless{}{-}} \ControlFlowTok{function}\NormalTok{(n)\{}
\NormalTok{  secuencia }\OtherTok{\textless{}{-}} \FunctionTok{sample}\NormalTok{(}\FunctionTok{c}\NormalTok{(}\StringTok{\textquotesingle{}A\textquotesingle{}}\NormalTok{,}\StringTok{\textquotesingle{}T\textquotesingle{}}\NormalTok{,}\StringTok{\textquotesingle{}G\textquotesingle{}}\NormalTok{,}\StringTok{\textquotesingle{}C\textquotesingle{}}\NormalTok{), n, }\AttributeTok{replace =} \ConstantTok{TRUE}\NormalTok{)}
\NormalTok{  vec\_secuencia }\OtherTok{\textless{}{-}} \FunctionTok{c}\NormalTok{(secuencia)}
  \FunctionTok{return}\NormalTok{(}\FunctionTok{c}\NormalTok{(}\FunctionTok{paste}\NormalTok{(secuencia, }\AttributeTok{collapse =} \StringTok{\textquotesingle{}\textquotesingle{}}\NormalTok{),n))}
\NormalTok{\}}
\CommentTok{\#TAMAÑO DE SECUENCIA!!}
\NormalTok{x }\OtherTok{\textless{}{-}} \FunctionTok{crearsecuenciaAdn}\NormalTok{(}\DecValTok{30}\NormalTok{)}
\NormalTok{x[}\DecValTok{2}\NormalTok{]}
\end{Highlighting}
\end{Shaded}

\begin{verbatim}
## [1] "30"
\end{verbatim}

\begin{enumerate}
\def\labelenumi{\arabic{enumi}.}
\setcounter{enumi}{1}
\tightlist
\item
  Codifica una función que calcula el tamaño de una secuencia de DNA.
\end{enumerate}

\begin{Shaded}
\begin{Highlighting}[]
\CommentTok{\#Es el segundo valor de la lista que regresa la funcion de crear Adn}
\NormalTok{x[}\DecValTok{2}\NormalTok{]}
\end{Highlighting}
\end{Shaded}

\begin{verbatim}
## [1] "30"
\end{verbatim}

\begin{enumerate}
\def\labelenumi{\arabic{enumi}.}
\setcounter{enumi}{2}
\tightlist
\item
  Crea una función que recibe una secuencia de DNA e imprime el
  porcentaje de cada base en la secuencia
\end{enumerate}

\begin{Shaded}
\begin{Highlighting}[]
\CommentTok{\#?str\_count}
\NormalTok{x[}\DecValTok{1}\NormalTok{]}
\end{Highlighting}
\end{Shaded}

\begin{verbatim}
## [1] "GTGCACAGTCCCTATCCATGTTTAAAGGCT"
\end{verbatim}

\begin{Shaded}
\begin{Highlighting}[]
\NormalTok{porcentajes }\OtherTok{\textless{}{-}} \ControlFlowTok{function}\NormalTok{(Adn)\{}
  \FunctionTok{return}\NormalTok{(}\FunctionTok{c}\NormalTok{(}\FunctionTok{str\_count}\NormalTok{(Adn, }\StringTok{\textquotesingle{}A\textquotesingle{}}\NormalTok{)}\SpecialCharTok{/}\FunctionTok{nchar}\NormalTok{(Adn)  ,}\FunctionTok{str\_count}\NormalTok{(Adn, }\StringTok{\textquotesingle{}T\textquotesingle{}}\NormalTok{)}\SpecialCharTok{/}\FunctionTok{nchar}\NormalTok{(Adn) ,}\FunctionTok{str\_count}\NormalTok{(Adn, }\StringTok{\textquotesingle{}C\textquotesingle{}}\NormalTok{)}\SpecialCharTok{/}\FunctionTok{nchar}\NormalTok{(Adn),}\FunctionTok{str\_count}\NormalTok{(Adn, }\StringTok{\textquotesingle{}G\textquotesingle{}}\NormalTok{)}\SpecialCharTok{/}\FunctionTok{nchar}\NormalTok{(Adn)))}
\NormalTok{\}}
\NormalTok{prc }\OtherTok{\textless{}{-}} \FunctionTok{porcentajes}\NormalTok{(x[}\DecValTok{1}\NormalTok{])}
\FunctionTok{paste}\NormalTok{(}\StringTok{"Porcentaje de A\textquotesingle{}s en la cadena:"}\NormalTok{, prc[}\DecValTok{1}\NormalTok{] }\SpecialCharTok{*}\DecValTok{100}\NormalTok{)}
\end{Highlighting}
\end{Shaded}

\begin{verbatim}
## [1] "Porcentaje de A's en la cadena: 23.3333333333333"
\end{verbatim}

\begin{Shaded}
\begin{Highlighting}[]
\FunctionTok{paste}\NormalTok{(}\StringTok{"Porcentaje de T\textquotesingle{}s en la cadena:"}\NormalTok{, prc[}\DecValTok{2}\NormalTok{] }\SpecialCharTok{*}\DecValTok{100}\NormalTok{)}
\end{Highlighting}
\end{Shaded}

\begin{verbatim}
## [1] "Porcentaje de T's en la cadena: 30"
\end{verbatim}

\begin{Shaded}
\begin{Highlighting}[]
\FunctionTok{paste}\NormalTok{(}\StringTok{"Porcentaje de C\textquotesingle{}s en la cadena:"}\NormalTok{, prc[}\DecValTok{3}\NormalTok{] }\SpecialCharTok{*}\DecValTok{100}\NormalTok{)}
\end{Highlighting}
\end{Shaded}

\begin{verbatim}
## [1] "Porcentaje de C's en la cadena: 26.6666666666667"
\end{verbatim}

\begin{Shaded}
\begin{Highlighting}[]
\FunctionTok{paste}\NormalTok{(}\StringTok{"Porcentaje de G\textquotesingle{}s en la cadena:"}\NormalTok{, prc[}\DecValTok{4}\NormalTok{] }\SpecialCharTok{*}\DecValTok{100}\NormalTok{)}
\end{Highlighting}
\end{Shaded}

\begin{verbatim}
## [1] "Porcentaje de G's en la cadena: 20"
\end{verbatim}

\begin{enumerate}
\def\labelenumi{\arabic{enumi}.}
\setcounter{enumi}{3}
\tightlist
\item
  Programa una función que transcribe DNA a RNA, usa la siguiente tabla:
\end{enumerate}

\begin{Shaded}
\begin{Highlighting}[]
\NormalTok{x[}\DecValTok{1}\NormalTok{]}
\end{Highlighting}
\end{Shaded}

\begin{verbatim}
## [1] "GTGCACAGTCCCTATCCATGTTTAAAGGCT"
\end{verbatim}

\begin{Shaded}
\begin{Highlighting}[]
\NormalTok{AdnToRna }\OtherTok{\textless{}{-}} \ControlFlowTok{function}\NormalTok{(Adn)\{}
  \FunctionTok{return}\NormalTok{(}\FunctionTok{str\_replace\_all}\NormalTok{(Adn,}\StringTok{\textquotesingle{}T\textquotesingle{}}\NormalTok{,}\StringTok{\textquotesingle{}U\textquotesingle{}}\NormalTok{))}
\NormalTok{\}}
\NormalTok{Rna }\OtherTok{\textless{}{-}} \FunctionTok{AdnToRna}\NormalTok{(x[}\DecValTok{1}\NormalTok{])}
\NormalTok{Rna}
\end{Highlighting}
\end{Shaded}

\begin{verbatim}
## [1] "GUGCACAGUCCCUAUCCAUGUUUAAAGGCU"
\end{verbatim}

\begin{enumerate}
\def\labelenumi{\arabic{enumi}.}
\setcounter{enumi}{4}
\tightlist
\item
  Crea una función que traduce una secuencia de RNA a una secuencia de
  proteínas.
\end{enumerate}

\begin{Shaded}
\begin{Highlighting}[]
\NormalTok{Triple }\OtherTok{\textless{}{-}} \FunctionTok{c}\NormalTok{(}\StringTok{"UUU"}\NormalTok{,}\StringTok{"UUC"}\NormalTok{,}\StringTok{"UUA"}\NormalTok{,}\StringTok{"UUG"}\NormalTok{,}\StringTok{"CUU"}\NormalTok{,}\StringTok{"CUC"}\NormalTok{,}\StringTok{"CUA"}\NormalTok{,}\StringTok{"CUG"}\NormalTok{,}\StringTok{"AUU"}\NormalTok{,}\StringTok{"AUC"}\NormalTok{,}\StringTok{"AUA"}\NormalTok{,}\StringTok{"AUG"}\NormalTok{,}\StringTok{"GUU"}\NormalTok{,}\StringTok{"GUC"}\NormalTok{,}\StringTok{"GUA"}\NormalTok{,}\StringTok{"GUG"}\NormalTok{,}\StringTok{"UCU"}\NormalTok{,}\StringTok{"UCC"}\NormalTok{,}\StringTok{"UCA"}\NormalTok{,}\StringTok{"UCG"}\NormalTok{,}\StringTok{"CCU"}\NormalTok{,}\StringTok{"CCC"}\NormalTok{,}\StringTok{"CCA"}\NormalTok{,}\StringTok{"CCG"}\NormalTok{,}\StringTok{"ACU"}\NormalTok{,}\StringTok{"ACC"}\NormalTok{,}\StringTok{"ACA"}\NormalTok{,}\StringTok{"ACG"}\NormalTok{,}\StringTok{"GCU"}\NormalTok{,}\StringTok{"GCC"}\NormalTok{,}\StringTok{"GCA"}\NormalTok{,}\StringTok{"GCG"}\NormalTok{,}\StringTok{"UAU"}\NormalTok{,}\StringTok{"UAC"}\NormalTok{,}\StringTok{"UAA"}\NormalTok{,}\StringTok{"UAG"}\NormalTok{,}\StringTok{"CAU"}\NormalTok{,}\StringTok{"CAC"}\NormalTok{,}\StringTok{"CAA"}\NormalTok{,}\StringTok{"CAG"}\NormalTok{,}\StringTok{"AAU"}\NormalTok{,}\StringTok{"AAC"}\NormalTok{,}\StringTok{"AAA"}\NormalTok{,}\StringTok{"AAG"}\NormalTok{,}\StringTok{"GAU"}\NormalTok{,}\StringTok{"GAC"}\NormalTok{,}\StringTok{"GAA"}\NormalTok{,}\StringTok{"GAG"}\NormalTok{,}\StringTok{"UGU"}\NormalTok{,}\StringTok{"UGC"}\NormalTok{,}\StringTok{"UGA"}\NormalTok{,}\StringTok{"UGG"}\NormalTok{,}\StringTok{"CGU"}\NormalTok{,}\StringTok{"CGC"}\NormalTok{,}\StringTok{"CGA"}\NormalTok{,}\StringTok{"CGG"}\NormalTok{,}\StringTok{"AGU"}\NormalTok{,}\StringTok{"AGC"}\NormalTok{,}\StringTok{"AGA"}\NormalTok{,}\StringTok{"AGG"}\NormalTok{,}\StringTok{"GGU"}\NormalTok{,}\StringTok{"GGC"}\NormalTok{,}\StringTok{"GGA"}\NormalTok{,}\StringTok{"GGG"}\NormalTok{)}
\NormalTok{protein }\OtherTok{\textless{}{-}} \FunctionTok{c}\NormalTok{(}\StringTok{"Phe"}\NormalTok{,}\StringTok{"Phe"}\NormalTok{,}\StringTok{"Leu"}\NormalTok{,}\StringTok{"Leu"}\NormalTok{,}\StringTok{"Leu"}\NormalTok{,}\StringTok{"Leu"}\NormalTok{,}\StringTok{"Leu"}\NormalTok{,}\StringTok{"Leu"}\NormalTok{,}\StringTok{"Ile"}\NormalTok{,}\StringTok{"Ile"}\NormalTok{,}\StringTok{"Ile"}\NormalTok{,}\StringTok{"Met"}\NormalTok{,}\StringTok{"Val"}\NormalTok{,}\StringTok{"Val"}\NormalTok{,}\StringTok{"Val"}\NormalTok{,}\StringTok{"Val"}\NormalTok{,}\StringTok{"Ser"}\NormalTok{,}\StringTok{"Ser"}\NormalTok{,}\StringTok{"Ser"}\NormalTok{,}\StringTok{"Ser"}\NormalTok{,}\StringTok{"Pro"}\NormalTok{,}\StringTok{"Pro"}\NormalTok{,}\StringTok{"Pro"}\NormalTok{,}\StringTok{"Pro"}\NormalTok{,}\StringTok{"Thr"}\NormalTok{,}\StringTok{"Thr"}\NormalTok{,}\StringTok{"Thr"}\NormalTok{,}\StringTok{"Thr"}\NormalTok{,}\StringTok{"Ala"}\NormalTok{,}\StringTok{"Ala"}\NormalTok{,}\StringTok{"Ala"}\NormalTok{,}\StringTok{"Ala"}\NormalTok{,}\StringTok{"Tyr"}\NormalTok{,}\StringTok{"Tyr"}\NormalTok{,}\StringTok{"Stop"}\NormalTok{,}\StringTok{"Stop"}\NormalTok{,}\StringTok{"His"}\NormalTok{,}\StringTok{"His"}\NormalTok{,}\StringTok{"Gln"}\NormalTok{,}\StringTok{"Gln"}\NormalTok{,}\StringTok{"Asn"}\NormalTok{,}\StringTok{"Asn"}\NormalTok{,}\StringTok{"Lys"}\NormalTok{,}\StringTok{"Lys"}\NormalTok{,}\StringTok{"Asp"}\NormalTok{,}\StringTok{"Asp"}\NormalTok{,}\StringTok{"Glu"}\NormalTok{,}\StringTok{"Glu"}\NormalTok{,}\StringTok{"Cys"}\NormalTok{,}\StringTok{"Cys"}\NormalTok{,}\StringTok{"Stop"}\NormalTok{,}\StringTok{"Trp"}\NormalTok{,}\StringTok{"Arg"}\NormalTok{,}\StringTok{"Arg"}\NormalTok{,}\StringTok{"Arg"}\NormalTok{,}\StringTok{"Arg"}\NormalTok{,}\StringTok{"Ser"}\NormalTok{,}\StringTok{"Ser"}\NormalTok{,}\StringTok{"Arg"}\NormalTok{,}\StringTok{"Arg"}\NormalTok{,}\StringTok{"Gly"}\NormalTok{,}\StringTok{"Gly"}\NormalTok{,}\StringTok{"Gly"}\NormalTok{,}\StringTok{"Gly"}\NormalTok{)}

\CommentTok{\#vector \textless{}{-} seq(from = 1, to = x[2], by = 3)}
\CommentTok{\#vector}
\CommentTok{\#strand \textless{}{-} str\_sub(Rna, vector, vector + 2)}
\CommentTok{\#strand}


\NormalTok{RnatoProtein }\OtherTok{\textless{}{-}} \ControlFlowTok{function}\NormalTok{(Rna)\{}
\NormalTok{  vector }\OtherTok{\textless{}{-}} \FunctionTok{seq}\NormalTok{(}\AttributeTok{from =} \DecValTok{1}\NormalTok{, }\AttributeTok{to =}\NormalTok{ x[}\DecValTok{2}\NormalTok{], }\AttributeTok{by =} \DecValTok{3}\NormalTok{)}
\NormalTok{  strand }\OtherTok{\textless{}{-}} \FunctionTok{str\_sub}\NormalTok{(Rna, vector, vector }\SpecialCharTok{+} \DecValTok{2}\NormalTok{)}
\NormalTok{  k }\OtherTok{\textless{}{-}} \FunctionTok{match}\NormalTok{(strand,Triple)}
  \FunctionTok{return}\NormalTok{(}\FunctionTok{paste}\NormalTok{(protein[k], }\AttributeTok{collapse =} \StringTok{\textquotesingle{}\textquotesingle{}}\NormalTok{))}
\NormalTok{\}}
\FunctionTok{RnatoProtein}\NormalTok{(Rna)}
\end{Highlighting}
\end{Shaded}

\begin{verbatim}
## [1] "ValHisSerProTyrProCysLeuLysAla"
\end{verbatim}

6.Escribe una función que recibe una hebra directa y regresa la hebra
inversa.

\begin{Shaded}
\begin{Highlighting}[]
\NormalTok{dnaToInverse }\OtherTok{\textless{}{-}}\ControlFlowTok{function}\NormalTok{ (dna)\{}
\NormalTok{  inverseDna }\OtherTok{\textless{}{-}}\StringTok{\textquotesingle{}\textquotesingle{}}
  \ControlFlowTok{for}\NormalTok{ (base }\ControlFlowTok{in} \DecValTok{1}\SpecialCharTok{:}\FunctionTok{nchar}\NormalTok{(dna))\{}
    \ControlFlowTok{if}\NormalTok{ (}\FunctionTok{substr}\NormalTok{(dna, base, base)}\SpecialCharTok{==} \StringTok{\textquotesingle{}T\textquotesingle{}}\NormalTok{)\{}
\NormalTok{      inverseDna }\OtherTok{=} \FunctionTok{paste}\NormalTok{(inverseDna,}\StringTok{\textquotesingle{}C\textquotesingle{}}\NormalTok{,}\AttributeTok{sep =} \StringTok{\textquotesingle{}\textquotesingle{}}\NormalTok{)}
\NormalTok{    \}}
    \ControlFlowTok{else} \ControlFlowTok{if}\NormalTok{ (}\FunctionTok{substr}\NormalTok{(dna, base, base)}\SpecialCharTok{==} \StringTok{\textquotesingle{}C\textquotesingle{}}\NormalTok{)\{}
\NormalTok{      inverseDna }\OtherTok{=} \FunctionTok{paste}\NormalTok{(inverseDna,}\StringTok{\textquotesingle{}T\textquotesingle{}}\NormalTok{, }\AttributeTok{sep =} \StringTok{\textquotesingle{}\textquotesingle{}}\NormalTok{)   }
\NormalTok{    \}}
   \ControlFlowTok{else} \ControlFlowTok{if}\NormalTok{ (}\FunctionTok{substr}\NormalTok{(dna, base, base)}\SpecialCharTok{==} \StringTok{\textquotesingle{}G\textquotesingle{}}\NormalTok{)\{}
\NormalTok{      inverseDna }\OtherTok{=} \FunctionTok{paste}\NormalTok{(inverseDna,}\StringTok{\textquotesingle{}A\textquotesingle{}}\NormalTok{, }\AttributeTok{sep=}\StringTok{\textquotesingle{}\textquotesingle{}}\NormalTok{)}
\NormalTok{    \}}
   \ControlFlowTok{else} \ControlFlowTok{if}\NormalTok{ (}\FunctionTok{substr}\NormalTok{(dna, base, base)}\SpecialCharTok{==} \StringTok{\textquotesingle{}A\textquotesingle{}}\NormalTok{)\{}
\NormalTok{      inverseDna }\OtherTok{=} \FunctionTok{paste}\NormalTok{(inverseDna,}\StringTok{\textquotesingle{}G\textquotesingle{}}\NormalTok{, }\AttributeTok{sep =} \StringTok{\textquotesingle{}\textquotesingle{}}\NormalTok{)   }
\NormalTok{    \}}
\NormalTok{  \}}
  \FunctionTok{return}\NormalTok{ (inverseDna)}
\NormalTok{\}}

\FunctionTok{dnaToInverse}\NormalTok{(x[}\DecValTok{1}\NormalTok{])}
\end{Highlighting}
\end{Shaded}

\begin{verbatim}
## [1] "ACATGTGACTTTCGCTTGCACCCGGGAATC"
\end{verbatim}

\begin{enumerate}
\def\labelenumi{\arabic{enumi}.}
\setcounter{enumi}{6}
\tightlist
\item
  Escribe una función qué recibe una hebra directa y obtiene la hebra
  complementaria
\end{enumerate}

\begin{Shaded}
\begin{Highlighting}[]
\NormalTok{dnaToComplementary }\OtherTok{\textless{}{-}}\ControlFlowTok{function}\NormalTok{ (dna)\{}
\NormalTok{  complementaryDna }\OtherTok{\textless{}{-}}\StringTok{\textquotesingle{}\textquotesingle{}}
  \ControlFlowTok{for}\NormalTok{ (base }\ControlFlowTok{in} \DecValTok{1}\SpecialCharTok{:}\FunctionTok{nchar}\NormalTok{(dna))\{}
    \ControlFlowTok{if}\NormalTok{ (}\FunctionTok{substr}\NormalTok{(dna, base, base)}\SpecialCharTok{==} \StringTok{\textquotesingle{}T\textquotesingle{}}\NormalTok{)\{}
\NormalTok{      complementaryDna }\OtherTok{=} \FunctionTok{paste}\NormalTok{(complementaryDna,}\StringTok{\textquotesingle{}A\textquotesingle{}}\NormalTok{,}\AttributeTok{sep =} \StringTok{\textquotesingle{}\textquotesingle{}}\NormalTok{)}
\NormalTok{    \}}
    \ControlFlowTok{else} \ControlFlowTok{if}\NormalTok{ (}\FunctionTok{substr}\NormalTok{(dna, base, base)}\SpecialCharTok{==} \StringTok{\textquotesingle{}C\textquotesingle{}}\NormalTok{)\{}
\NormalTok{      complementaryDna }\OtherTok{=} \FunctionTok{paste}\NormalTok{(complementaryDna,}\StringTok{\textquotesingle{}G\textquotesingle{}}\NormalTok{, }\AttributeTok{sep =} \StringTok{\textquotesingle{}\textquotesingle{}}\NormalTok{)   }
\NormalTok{    \}}
   \ControlFlowTok{else} \ControlFlowTok{if}\NormalTok{ (}\FunctionTok{substr}\NormalTok{(dna, base, base)}\SpecialCharTok{==} \StringTok{\textquotesingle{}G\textquotesingle{}}\NormalTok{)\{}
\NormalTok{      complementaryDna }\OtherTok{=} \FunctionTok{paste}\NormalTok{(complementaryDna,}\StringTok{\textquotesingle{}C\textquotesingle{}}\NormalTok{, }\AttributeTok{sep=}\StringTok{\textquotesingle{}\textquotesingle{}}\NormalTok{)}
\NormalTok{    \}}
   \ControlFlowTok{else} \ControlFlowTok{if}\NormalTok{ (}\FunctionTok{substr}\NormalTok{(dna, base, base)}\SpecialCharTok{==} \StringTok{\textquotesingle{}A\textquotesingle{}}\NormalTok{)\{}
\NormalTok{      complementaryDna }\OtherTok{=} \FunctionTok{paste}\NormalTok{(complementaryDna,}\StringTok{\textquotesingle{}T\textquotesingle{}}\NormalTok{, }\AttributeTok{sep =} \StringTok{\textquotesingle{}\textquotesingle{}}\NormalTok{)   }
\NormalTok{    \}}
\NormalTok{  \}}
  \FunctionTok{return}\NormalTok{ (complementaryDna)}
\NormalTok{\}}
\NormalTok{x[}\DecValTok{1}\NormalTok{]}
\end{Highlighting}
\end{Shaded}

\begin{verbatim}
## [1] "GTGCACAGTCCCTATCCATGTTTAAAGGCT"
\end{verbatim}

\begin{Shaded}
\begin{Highlighting}[]
\NormalTok{complementaryDna }\OtherTok{\textless{}{-}} \FunctionTok{dnaToComplementary}\NormalTok{(x[}\DecValTok{1}\NormalTok{])}
\NormalTok{complementaryDna}
\end{Highlighting}
\end{Shaded}

\begin{verbatim}
## [1] "CACGTGTCAGGGATAGGTACAAATTTCCGA"
\end{verbatim}

\begin{enumerate}
\def\labelenumi{\arabic{enumi}.}
\setcounter{enumi}{7}
\tightlist
\item
  Si escribimos la secuencia de la hebra complementaria en sentido
  inverso (5'→3') se obtiene la complementaria inversa
  (reverse-complement):
\end{enumerate}

\begin{enumerate}
\def\labelenumi{\alph{enumi}.}
\tightlist
\item
  Hebra complementaria: 3'-ACGCTATG-5'
\item
  Hebra complementaria inversa: 5'-GTATCGCA-3' Escribe la función en R
  para obtener la hebra complementaria inversa, desde una hebra
  complementaria.
\end{enumerate}

\begin{Shaded}
\begin{Highlighting}[]
\NormalTok{reverse\_complement }\OtherTok{\textless{}{-}} \FunctionTok{stri\_reverse}\NormalTok{(complementaryDna)}
\NormalTok{reverse\_complement}
\end{Highlighting}
\end{Shaded}

\begin{verbatim}
## [1] "AGCCTTTAAACATGGATAGGGACTGTGCAC"
\end{verbatim}

\begin{enumerate}
\def\labelenumi{\arabic{enumi}.}
\setcounter{enumi}{8}
\tightlist
\item
  Prueba cada una de las funciones y recuerda que las secuencias pueden
  contener caracteres especiales donde aparte de ATGC, en ADN, y AUGC,
  en ARN, pudimos ver algunos guiones (omitidos) y N (desconocido)
  nucleótido.
\end{enumerate}

\begin{Shaded}
\begin{Highlighting}[]
\NormalTok{Adn }\OtherTok{\textless{}{-}} \FunctionTok{crearsecuenciaAdn}\NormalTok{(}\DecValTok{60}\NormalTok{)}
\NormalTok{Adn[}\DecValTok{1}\NormalTok{]}
\end{Highlighting}
\end{Shaded}

\begin{verbatim}
## [1] "CTGCATATTGAATTTCTGTGTTAGAGGGCTAGTAAACTAAGAGAATTCGCCACGATGGCG"
\end{verbatim}

\begin{Shaded}
\begin{Highlighting}[]
\NormalTok{Adn[}\DecValTok{2}\NormalTok{]}
\end{Highlighting}
\end{Shaded}

\begin{verbatim}
## [1] "60"
\end{verbatim}

\begin{Shaded}
\begin{Highlighting}[]
\NormalTok{prc }\OtherTok{\textless{}{-}} \FunctionTok{porcentajes}\NormalTok{(Adn[}\DecValTok{1}\NormalTok{])}
\FunctionTok{paste}\NormalTok{(}\StringTok{"Porcentaje de A\textquotesingle{}s en la cadena:"}\NormalTok{, prc[}\DecValTok{1}\NormalTok{] }\SpecialCharTok{*}\DecValTok{100}\NormalTok{)}
\end{Highlighting}
\end{Shaded}

\begin{verbatim}
## [1] "Porcentaje de A's en la cadena: 28.3333333333333"
\end{verbatim}

\begin{Shaded}
\begin{Highlighting}[]
\FunctionTok{paste}\NormalTok{(}\StringTok{"Porcentaje de T\textquotesingle{}s en la cadena:"}\NormalTok{, prc[}\DecValTok{2}\NormalTok{] }\SpecialCharTok{*}\DecValTok{100}\NormalTok{)}
\end{Highlighting}
\end{Shaded}

\begin{verbatim}
## [1] "Porcentaje de T's en la cadena: 28.3333333333333"
\end{verbatim}

\begin{Shaded}
\begin{Highlighting}[]
\FunctionTok{paste}\NormalTok{(}\StringTok{"Porcentaje de C\textquotesingle{}s en la cadena:"}\NormalTok{, prc[}\DecValTok{3}\NormalTok{] }\SpecialCharTok{*}\DecValTok{100}\NormalTok{)}
\end{Highlighting}
\end{Shaded}

\begin{verbatim}
## [1] "Porcentaje de C's en la cadena: 16.6666666666667"
\end{verbatim}

\begin{Shaded}
\begin{Highlighting}[]
\FunctionTok{paste}\NormalTok{(}\StringTok{"Porcentaje de G\textquotesingle{}s en la cadena:"}\NormalTok{, prc[}\DecValTok{4}\NormalTok{] }\SpecialCharTok{*}\DecValTok{100}\NormalTok{)}
\end{Highlighting}
\end{Shaded}

\begin{verbatim}
## [1] "Porcentaje de G's en la cadena: 26.6666666666667"
\end{verbatim}

\begin{Shaded}
\begin{Highlighting}[]
\NormalTok{Rna }\OtherTok{\textless{}{-}} \FunctionTok{AdnToRna}\NormalTok{(Adn[}\DecValTok{1}\NormalTok{])}
\NormalTok{Rna}
\end{Highlighting}
\end{Shaded}

\begin{verbatim}
## [1] "CUGCAUAUUGAAUUUCUGUGUUAGAGGGCUAGUAAACUAAGAGAAUUCGCCACGAUGGCG"
\end{verbatim}

\begin{Shaded}
\begin{Highlighting}[]
\NormalTok{ProteinRna }\OtherTok{\textless{}{-}} \FunctionTok{RnatoProtein}\NormalTok{(Rna)}
\NormalTok{ProteinRna}
\end{Highlighting}
\end{Shaded}

\begin{verbatim}
## [1] "LeuHisIleGluPheLeuCysStopArgAla"
\end{verbatim}

\begin{Shaded}
\begin{Highlighting}[]
\NormalTok{Inversa }\OtherTok{\textless{}{-}} \FunctionTok{dnaToInverse}\NormalTok{(Adn[}\DecValTok{1}\NormalTok{])}
\NormalTok{Inversa}
\end{Highlighting}
\end{Shaded}

\begin{verbatim}
## [1] "TCATGCGCCAGGCCCTCACACCGAGAAATCGACGGGTCGGAGAGGCCTATTGTAGCAATA"
\end{verbatim}

\begin{Shaded}
\begin{Highlighting}[]
\NormalTok{complementaryDna }\OtherTok{\textless{}{-}} \FunctionTok{dnaToComplementary}\NormalTok{(Adn[}\DecValTok{1}\NormalTok{])}
\NormalTok{complementaryDna}
\end{Highlighting}
\end{Shaded}

\begin{verbatim}
## [1] "GACGTATAACTTAAAGACACAATCTCCCGATCATTTGATTCTCTTAAGCGGTGCTACCGC"
\end{verbatim}

\end{document}
